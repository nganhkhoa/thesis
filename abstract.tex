\clearpage
% \addcontentsline{toc}{section}{\abstractname}
\begin{abstract}

Digital forensics is an action taken to extract raw data from computers and analyze that data to discover information or to learn about historical events that happened to the target machine. Computer forensics is more and more popular today where cyberattacks happen almost every day. When cyberattacks happen, we must stop the threat, but gathering information about the attack also has the same degree of importance. It is surprising how many information one can learn given only a snapshot of the memory, from running processes to hidden files and cryptographic keys. With high demands, people have come up with many techniques to analyze disk images, hard drives, and physical memory. No matter how great techniques we have for digital forensics, it is still worse than preventing an attack from happening. Even though Antivirus software, live reporting system and firewall mitigate some known attacks but with the new arising group of malware operate in stealth mode, like Cryptomining malware, or Rootkits remain a significant threat to organizations. In this paper, we would like to introduce a scanning system based on digital forensics to list running processes and to assist in the detection of malware.

\end{abstract}
\clearpage

\section[Result]{Result}

We have written a few utilities to demonstrate what we can do with LPUS. The
utilities are pool tag scanning for processes, threads, drivers, files, kernel
modules; traversing the list of processes from \texttt{PsActiveProcessHead},
\texttt{KiProcessListHead}, \texttt{HandleTableListHead}; traversing the list
of loaded kernel module from \texttt{PsLoadedModuleList}; list out the unloaded
drivers, and the SSDT table; and inspecting any driver's 28 major functions.
We have also written a command similar to \texttt{psxview} of Volatility to
compare the results of processes found by many methods.

Although currently LPUS is only tested in Windows 7 and Windows 10 in
VirtualBox, it gives positive results. LPUS run and output the scanned objects
without crashing the systems. The results are checked alongside with WinDbg, a
Windows debugger, attached to the machine in virtualization. The test perform
as follows, LPUS runs and out the result, then the results are checked by
stoping the machine and attaching WinDbg. In WinDbg, commands such as
\texttt{dt} can be used to output the detail of a structure at an address, we
checked for every object found by LPUS with WinDbg. For list traversing, we
used the command in Listing.\ref{} to get the list of address in global lists.
The SSDT and driver 28 major functions is not checked through WinDbg, we
believe the addresses should be in kernel modules, which upon checking all
addresses output belongs to kernel modules. We also conducted a memory
acquisition using VirtualBox and WinPmem and compare the results found with
Volatility, the acquisition process happens seconds after the scans took place.
The results between LPUS, WinDbg and Volatility by VirtualBox memory
acquisition and Volatility by WinPmem is provided in Table.\ref{}.

% TODO: insert test result table

\section[Memory Forensics]{Memory Forensics}

With all the preventions, malware still finds a way to bypass the security
setup and infect the system. When such an event occurs, a skilled technician
will begin analyzing the system. To start the investigation, they would need to
take the physical memory content of the system and analyze the content looking
for prominent malware. This is called memory forensics, a sophisticated process
done by a team of security researchers when a breach occurs or when an unknown
factor compromises the system.

\subsection[Memory acquisition]{Memory acquisition}

The first step in memory forensics is capturing the content of physical memory.
This step is often called as \textit{memory acquisition}. The goal was to load
the RAM data inside a file. There are multiple ways \cite{memoryAcquisition} to
do this in a running machine using a kernel driver:

\begin{enumerate}
\item Read \texttt{\textbackslash\textbackslash.\textbackslash PhysicalMemory}
\item Use \texttt{MmMapMemoryDumpMdl}
\item Use \texttt{MmMapIoSpace}
\item PTE remapping
\end{enumerate}

The RAM is a device, and Windows exposes the device as a file with the absolute
path \texttt{\textbackslash\textbackslash.\textbackslash PhysicalMemory}. By
reading this file, we are accessing the RAM itself. However, Windows prevented
reading this file. The second and third method maps physical page(s) to the
kernel space, the content is sent back to the user space to write out the file.
The fourth method is a technique described by Michael Cohen and Johannes
Stüttgen \cite{pteremap} to capture the memory using the PTE, page table entry.

Most memory acquisition tools are closed source, the one open-source memory
acquisition tool is WinPmem \cite{winpmem} and using the third and fourth
method to capture the RAM. WinPmem's author also encourages using the PTE
remapping method for better results. Other programs often used for memory
acquisition are \texttt{DumpIt} by Moonsol, \texttt{FTK Imager} by AccessData,
\texttt{Belkasoft RAM Capturer} by Belkasoft.

Hypervisor, VirtualBox or VMWare, can generate the memory file.  Although the
memory file format is different from each virtual machine implementation, the
documentation of the file format is available online.

The last way to acquire the memory content is through Windows debug mechanism.
Windows has an option to generate the memory file when the system crash, if
this option is on, Windows will produce the file at \texttt{C:\textbackslash
Windows\textbackslash MEMORY.DMP} on system crash. Windows also has multiple
option to the memory file, such as the \textit{full dump}, \textit{mini dump},
\textit{kernel dump}. These files are not officially documented by Windows and
intended to be used by Windows debugger, but researchers have reversed the file
format for forensics purposes \cite{mdmpfile} \cite{dmpfile}.

\subsection[KDBG]{KDBG}

KDBG is a Windows structure for debugging purposes, this structure contains
pointers to many kernel essential components, such as the list of processes.
Finding this structure in the memory file will give some insight into the
system.  This structure can be found using the four-byte ``KDBG'' in Windows
versions before 8. In Windows 8 and above, the structure is encoded and makes
it more challenging to find the structure. However, by brute-scanning and
brute-decoding \cite{kdbgEncoded}, one can still find the structure.

\subsection[Pool tag scanning]{Pool tag scanning}

As described in Section \ref{sec:pool}, each block has a pool header contains a
four-byte tag. By scanning for a specific tag value that the OS uses to
allocate for a kernel structure, we can search for all the object allocated,
and sometimes de-allocated but not overwritten. Each tag-block matches should
be further checked for validity, the pool body size (block size subtract the
pool header size) should be bigger than the object holding (if the object is an
executive object), or equal to the object.  Optional checks for pool type value
are done with objects usually allocated in one pool section.  E.g.,
\texttt{\_EPROCESS} is always allocated inside the non-paged pool section,
hence the pool type is 2. This technique is called \textit{Pool tag scanning}
\cite{pooltagscan}, and often done by a full search in the memory file. In
2016, Sylve, Joe T and Marziale, Vico and Richard III, Golden G
\cite{sylve2016pool} improved this technique by locating the range of the
non-paged pool region and scan only the pages that are valid. They named the
technique \textit{Pool tag quick scanning}.

The pool tag quick scanning research is conducted on Windows Vista and 8, we
extended the findings of these ranges on Windows XP, 7, 8, and 10 (all versions
from 2015 to 2020), with reference from the Rekall project. The result can be
found in Table \ref{tab:nonpaged}.

\begin{table}[]
\begin{tabular}{lll}
\hline
Windows version & Pool start                      & Pool end                      \\ \hline
XP              & \texttt{nt!MmNonPagedPoolStart} & \texttt{nt!MmNonPagedPoolEnd} \\
7, 8, 8.1       & \texttt{nt!MiNonPagedPoolStart} & \texttt{nt!MiNonPagedPoolEnd} \\
10 2015         & \texttt{nt!MiState}
                & \texttt{nt!MiState} \\
                & \texttt{->SystemNodeInformation}
                & \texttt{->SystemNodeInformation} \\
                & \texttt{.NonPagedPoolFirstVa}
                & \texttt{.NonPagedPoolLastVa} \\
10 2016-2019    & \texttt{nt!MiState}
                & \texttt{nt!MiState} \\
                & \texttt{->Hardware}
                & \texttt{->Hardware} \\
                & \texttt{.SystemNodeInformation}
                & \texttt{.SystemNodeInformation} \\
                & \texttt{.NonPagedPoolFirstVa}
                & \texttt{.NonPagedPoolLastVa} \\
10 2020         & \texttt{nt!MiState}
                & \texttt{nt!MiState} \\
                & \texttt{->Hardware}
                & \texttt{->Hardware} \\
                & \texttt{.SystemNodeNonPagedPool}
                & \texttt{.SystemNodeNonPagedPool} \\
                & \texttt{.NonPagedPoolFirstVa}
                & \texttt{.NonPagedPoolLastVa} \\ \hline
\end{tabular}
  {\raggedright \texttt{nt!foo} is the address of \texttt{foo} in kernel space
    \par}
  {\raggedright \texttt{nt!MiState} is a pointer to the type
  \texttt{\_MI\_SYSTEM\_INFORMATION}, this structure layout is different with
  each Windows 10 releases \par}
  {\raggedright \texttt{nt!MiNonPagedPoolStart} or
  \texttt{nt!MiNonPagedPoolStartAligned} both valid for Windows 7, 8, 8.1. Aligned
  version points to the first valid page in the pool \par}

  \caption{Non paged pool start and end}
  \label{tab:nonpaged}
\end{table}

Pool tag scanning is often used to ensure there is no process hiding from the
kernel list by simple removal. Comparing the pool tag scanning result with the
list traversing, we may be able to reveal unlinked objects from the global
list. However, pool tag scanning can be used to reveal other objects often not
reside on any list. These can be internet-related structures, kernel callbacks.
Although Windows does not document this structure, many people have studied the
Windows source code (by reversing) and came up with structure definitions and
reasonable names. In Table \ref{tab:pooltag}, we list the tags and the
structure where pool tag scanning is often used to find. These tags are used in
the Volatility framework.

\begin{table}[]
\begin{tabular}{llll}
\hline
  Structure                 & Structure Name                     & Pool Tag (new) & Pool Tag (old)        \\ \hline
  Driver Object             & \texttt{\_DRIVER\_OBJECT}          & Driv           & Dri\textbackslash xf6 \\
  File Object               & \texttt{\_FILE\_OBJECT}            & File           & Fil\textbackslash xe5 \\
  Process                   & \texttt{\_EPROCESS}                & Proc           & Pro\textbackslash xe3 \\
  Thread                    & \texttt{\_ETHREAD}                 & Thre           & Thr\textbackslash xe5 \\
  Modules                   & \texttt{\_LDR\_DATA\_TABLE\_ENTRY} & MmLd           &                       \\
  TCP endpoint              & *\texttt{\_TCP\_ENDPOINT}          & TcpE           &                       \\
  TCP listener              & *\texttt{\_TCP\_LISTENER}          & TcpL           &                       \\
  UDP endpoint              & *\texttt{\_UDP\_ENDPOINT}          & UdpA           &                       \\
  TCP connection            & *\texttt{\_TCPT\_OBJECT}           & TCPT           &                       \\
  Notification Packet       & *\texttt{\_NOTIFICATION\_PACKET}   & IoFs           &                       \\
  Shutdown Packet           & *\texttt{\_SHUTDOWN\_PACKET}       & IoSh           &                       \\
  Generic callback          & *\texttt{\_GENERIC\_CALLBACK}      & Cbrb           &                       \\
  DbgPrint callback         & *\texttt{\_DBGPRINT\_CALLBACK}     & DbCb           &                       \\
  Registry callback         & *\texttt{\_REGISTRY\_CALLBACK}     & CMcb           &                       \\
  Hardware profile callback & *\texttt{\_NOTIFY\_ENTRY\_HEADER}  & Pnp9           &                       \\
  Device interface callback & *\texttt{\_NOTIFY\_ENTRY\_HEADER}  & PnpD           &                       \\
  Target device callback    & *\texttt{\_NOTIFY\_ENTRY\_HEADER}  & PnpC           &                       \\
\hline
\end{tabular}
  {\raggedright *: This structure is not documented by Windows, the name is taken
  from Volatility framework \par}
  \caption{Comprehensive list of pool tag and corespondent structure}
  \label{tab:pooltag}
\end{table}

\subsection[SSDT and IRP hook detection]{SSDT and IRP hook detection}

SSDT hooking is dangerous in a way it can modify the kernel system call
behavior.  Detection of SSDT involves finding the system table and locate the
function address between loaded modules.The SSDT is located at
\texttt{KiServiceTable}, with \texttt{KiServiceLimit} elements of 32-bit value.
Every element is an offset to the function of the SSDT. To retrieve the address
of the function, we follow Listing \ref{lst:ssdt}.


\begin{lstlisting}[language=cpp,caption={Retrieve SSDT functions},label={lst:ssdt},float,floatplacement=H]
int32_t* KiServiceTable;
int KiServiceLimit;
void* ssdt[KiServiceLimit];
for (int i = 0; i < KiServiceLimit; i++) {
  ssdt[i] = (void*) ((int64_t)KiServiceTable + (KiServiceTable[i] >> 4));
}
\end{lstlisting}

After recovering all the functions in the system call table, we will check
every loaded module of type \texttt{\_KLDR\_DATA\_TABLE\_ENTRY} and see if the
address of the function is in the range of the \texttt{DllBase} and
\texttt{DllBase + Size}, if the address is between this range, the function is
in this module.  Because these are system calls, the functions should reside in
the kernel module, \texttt{ntoskrnl.exe}. If some functions reside in another
module, the SSDT probably hooked. Some malware even modify the function
instruction to jump or call to itself, this can be check by dumping several
bytes at the function address and decompile into machine code checking for jump
instruction and check the referenced address module.

IRP hooking is in driver 28 major functions. To detect this type of hook, we
list the 28 major functions searching for the loaded modules containing these
functions. It should be noted that the driver should either have the IRP
functions belongs to the driver or the kernel modules. If the functions are of
another module, then it was probably hooked. If the malware patch the
underlying function to jump or call itself, decompile the first few bytes
looking for such instruction and check the referenced address' module. For
the advanced IRP hooking method, this can only be detected after reversing the
function being hooked.

% To make the result easier to read and understand, the loaded module are parsed
% to get the exported symbols. When printing out the result, the module name with
% the function name will be printed out.

\subsection[Unloaded kernel drivers]{Unloaded kernel drivers}

There are some cases where malware load and unload the kernel driver almost
immediately (Rustock). In these scenarios, looking inside the unloaded driver
list might shed some light on this type of malware. Windows stores a list of
unloaded kernel drivers, the pointer to the list is located at
\texttt{MmUnloadedDrivers}. The list contains at most 50 members sized 0x28
bytes, the first 0x18 bytes in each member is of type
\texttt{\_UNLOADED\_DRIVER}.  \texttt{\_UNLOADED\_DRIVER} only contains the
name of the driver and the start and end address of the driver, the
\texttt{\_DRIVER\_OBJECT}, in most cases, is deallocated. If lucky, pool tag
scanning may show drivers unloaded, but it is usually not the case.

\chapter[Introduction]{Introduction}

In this chapter, we explain the definitions along with the current state of digital forensics and some contexts of today computer system and the current state we are in.

\section[Motivation]{Motivation}

Throughout the years of computer development, computers have become a standard method for humans around the world to study, work and entertain. Most activities of our daily lives involve computers. Individuals across the globe have created systems running on computers to assist them in doing common and complex tasks. However, this somehow influenced other people to commit harmful activities, these people often refer to as \textit{hacker}. Hackers have been creating software to cause damage and steal confidential or private information, these software are malware. Furthermore, lately along with the trend of cryptocurrency, while ordinary people commit their investment by using crypto mining machines, hackers, on the other hand, create sophisticated software (cryptojacking malware) that stealthily installed on a victim machine and mine cryptocurrency without the victim's acknowledgement.

Security researchers have been struggling to find ways to mitigate the gaining rate of attacks. However, it was never close to perfection. Security researchers uses file checksum and unique bytes sequences in file to create a file signature and combines a list of malware signature to a database, for example yararules \cite{yararules}. Relying on file signature database for filtering file often miss out new one, thus, it is highly vulnerable to the newer class of malware. To counterattack these new malware when an attack happens, digital forensics is performed. Digital forensics which as described by The Forensics Research Workshop I \cite{roadmap}:

\say{The use of scientifically derived and proven methods toward the preservation, collection, validation, identification, analysis, interpretation, documentation and presentation of digital evidence derived from digital sources for the purpose of facilitating or furthering the reconstruction of events found to be criminal, or helping to anticipate unauthorized actions shown to be disruptive to planned operations.}

Digital forensics includes many different aspects, however, the most intrigued part of digital forensics is memory forensics, which "provides unprecedented visibility into the runtime state of the system, such as which processes were running, open network connections, and recently executed commands" as stated in the book The Art of Memory Forensics \cite{ligh2014art}. Memory forensics provide a frame of the computer state, from which one can extract files and processes. Researchers often extract suspicious files from these memory samples, then reverse engineer it, if the file is a malware, they will create the file signature and add to the database. Digital forensics and memory forensics is very important to gain insight on the cyber attacks made by hackers. However, prevention is always better than fixing, and with the rise of the more steathy malware, the computer security industry is facing a major problem.

Malware are good pieces of software, but only before the disclosure. Therefore hackers creates malware hidden from the Task Manager and other process listing tools. From the early days of 1990s, they have been improving ways to hide malware. In 2017, a small report \cite{evolutionHidding} have revised on hiding techniques in malware over the years in the Windows OS. Most of these techniques is either preventable or mitigable. However, even in 2019, we can still observe incidences where malware hide itself so effectively. For example, Android malware that hid themselves on user mobile and was only found September 2019 after 5 months on the Google Play Store \cite{hiddenMalwareAndroid}, or the \textit{Titanium} backdoor\footnote{Programs that receives remote connections} on Windows 10 disclosed November 2019 \cite{titanium}, or the macOS malware, named \textit{unioncrypto}, that downloads and runs a hidden process in memory to mine cryptocurrency was only discovered December 2019 \cite{unioncrypto}. Researchers collected and analyzed these malware, but a normal user would not be able to do that. For a normal user, if the anti-virus software failed to flag the file as mallicious, his computer will be infected. To know whether a system is having a hidden malware running, one must send an investigator a memory extraction. Such process is complex, long and costly, a user must know how to extract the memory and hire an investigator to analyze. If there were a tool to do memory forensics live, while the system is running, and extract the hidden malware automatically, it would be easier for a normal user. Looking in live memory forensics, Shuaibur Rahman and Khan \cite{reviewLive} has a review of live forensics analysis techniques in 2015, and in the review there is one work that can extract running, finished, and cached processes \cite{comparativeLive}. However, the work is only restricted to the Linux OS and requires an experienced invetigator to do. Besides that, there are projects that implements memory forensics techniques but is limited to dump file analysis. We have researched for live memory analysis and have come up with a method for finding hidden processes.

Within the scope of the thesis, we will create a tool finding hidden processes aims at normal users. The tool could also submit the binary along with the process memory to the remote server. And with the market share of Windows over others OS is more than 75\% \cite{osMarketShare}, within the Windows OS, version 10 is now dominating with more than 60\% \cite{windowsShare}, this tool will focus on Windows 10 live memory analysis.

\section[Objectives]{Objectives}

In the scope of this thesis, we wish to:

\begin{itemize}
  \item Understand the basics concept of OS that supports memory forensics.
  \item Understand to some extend the internal of Windows operating system.
  \item Understand some techniques often used to do Windows memory forensics.
  \item Analyze some already existed tools doing memory forensics.
  \item Propose a method to find hidden running processes in a running Windows machine.
\end{itemize}

\section[Structure]{Structure}

% TODO

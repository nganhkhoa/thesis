\section[Rekall]{Rekall}

Rekall open-source project by Google \cite{rekall} is the second choice in
memory forensics.  Rekall is also written in Python 2 and provides almost the
same functionalities that Volatility has. Rekall uses the debug symbols to
resolve the structures layout and global variable offsets in Windows analysis.
One withdrawal from Rekall is probably the installation, where Rekall is harder
to install and use than Volatility.

Rekall has a memory capture module called WinPmem, which is also a kernel
driver for doing live forensics. On doing live memory forensics, WinPmem
creates a device at \texttt{\textbackslash\textbackslash .\textbackslash pmem}
and Rekall will perform the forensics on the device.

Rekall performs live forensics by mapping every physical memory pages and binds
the read operation to \texttt{\textbackslash\textbackslash .\textbackslash
pmem} in \texttt{IRP\_MJ\_READ}. When the user application read the file at an
offset, WinPmem driver will map the required physical page and copy to the
output buffer.
